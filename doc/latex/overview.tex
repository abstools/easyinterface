{% -*- mode: LaTeX; TeX-PDF-mode: t; TeX-master: "manual"; -*-
}


\chapter{Overview of the \ei Framework}
\label{ch:overview}

The \ei framework provides a simple way to build interfaces, e.g., a
web-interface or an Eclipse plugin, for tools written in (almost) any
programming language.
%
Moreover, it does not require the programmer to be familiar with any
GUI library or web programming. Roughly, the only requirement is that
the application can be executed from a command-line and that its
output goes to the standard output.
%

The goal of \ei is to provide developers with a toolkit to \emph{build
  their applications once and get several interfaces for free}.
%
\ei was originally developed for building a common frontend for
program analysis tools developed in the
\envisage\footnote{http://www.envisage-project.eu} project. This is
why, as the reader will notice later, its graphical user interfaces
are basically developing environments that allow editing programs,
etc.

In the rest of this chapter we overview the different components of
\ei, and explain how they are combined to achieve the above goal.

\begin{figure}[h]
\hrule\smallskip
\begin{center}
\includegraphics[width=0.5\textwidth]{fig/ei.pdf}
\end{center}
\caption{The Architecture of the \ei Framework}
\label{fig:eiframework}
\hrule
\end{figure}

\section{The Architecture of \ei}
\label{ch:overview:arch}

The architecture of \ei is depicted in
Figure~\ref{fig:eiframework}. It includes two main components:
%
(1) \emph{server side}: a machine with several applications (the
circles \texttt{App1}, \texttt{App2}, etc., in
Figure~\ref{fig:eiframework}) that can be executed from a command-line
and their output goes to the standard output. These are the
applications that we want to make available for the outside world,
i.e., execute them as services on the internet; and
%
(2) \emph{client side}: several clients that make it easy to
communicate with the server side to execute an application, etc.
%
In what follows, we first explain the inner components of the server
side and which problems they solve, and then we explain the client
side.
%

\section{The Server Side}
\label{ch:overview:arch:server}

The problem that we want to solve at the server side is: 
%
\begin{quote}
  Provide a uniform way for remotely accessing locally installed
  applications as services.
\end{quote}
%
This problem is solved by the \ei server, which is collection of PHP
programs that run on top of an HTTP server. This server allows
specifying how a local application can be executed and which
parameters it takes using simple configuration files
(\texttt{App1.cfg}, \texttt{App2.cfg}, etc.,
Figure~\ref{fig:eiframework}). For example, the following is a snippet
of such configuration file:

\medskip
\begin{lstlisting}
<app id="myapp" visible="true">
  ...
  <execinfo method="cmdline">
    <cmdlineapp>(*/path-to/myapp.sh*) _ei_parameters</cmdlineapp>
  </execinfo>
  <parameters prefix = "-" check="false">
    ...
    <selectone name="c">
      <option value="1" />
      <option value="2" />
    </selectone>
  </parameters>
</app>
\end{lstlisting}

\medskip
\noindent
This XML defines an application that has a unique identifier
\lst{myapp}.  The \lst{cmdlineapp} tag is a template that describes
how to execute the application from a command-line. Here
\lst{_ei_parameters} is a template parameter that will be replaced by
an appropriate value. The \lst{parameters} tag includes a list of
parameters accepted by the application. For example, there is a
parameter called ``\texttt{c}'' that can take one of the values $1$ or
$2$.
%
Once the configuration file is installed on the \ei server, anyone can
access the application using an HTTP POST request that includes the
following text:

\medskip
\begin{lstlisting}
{
  (*command: "execute",*)
  (*app\_id: "myapp",*)
  (*parameters:*) {
     (*c: ["1"],*)
     (*...*)
  },
  (*...*)
}
\end{lstlisting} 

\medskip
\noindent
When the \ei server receives such a request, it generates a
corresponding command-line (according to what is specified in the
configuration file), executes it, and redirect the standard output
back to the client.
%

\section{The Client Side}
\label{ch:overview:arch:client}

\begin{figure}[t]
\hrule\smallskip
\begin{center}
\includegraphics[width=1\textwidth]{fig/webclient.pdf}
\end{center}
\caption{\ei Web Client}
\label{fig:webclient}
\hrule
\end{figure}

Although we now have a relatively easy way to execute applications on
the server side, it is still not as easy as we aimed at.
%
Our aim is to simplify this process further by providing (graphical)
user interfaces that automatically (1) connect to the \ei server and
ask for the list of available applications; (2) let the user choose an
application to execute and set the values of the corresponding
parameters; (3) generate a corresponding request and send it to the
\ei server; and (4) shows the returned output to the user.
%
The \ei framework provides three such interfaces: a
\emph{web-interface} that can be executed in a browser and looks like
a developing environment (see Figure~\ref{fig:webclient}); an
Eclipse-plugin that runs within the Eclipse IDE; and a remote-shell
that can be used from a command-line.

Since the web-client and the Eclipse plugin are GUI based developing
environments, \ei provide also, to an application, the possibility to
generate output that has some graphical effects, e.g., open
dialog-boxes, highlight code lines, add markers, etc. To use this
feature, the applications should be modified to use the \ei output
language. The following is a snippet of such output:

\medskip
\begin{lstlisting}
<highlightlines dest="/Examples_1/iterative/sum.s"> 
  <lines> <line from="5" to="10"/> </lines>
</highlightlines>
...
<oncodelineclick dest="/Examples_1/iterative/sum.c" outclass="info" >
  <lines><line from="17" /></lines>
  <eicommands>
    <dialogbox boxtitle="Hey!"> 
      <content format="text">
       (* Click on the marker again to close this window *)
      </content>
    </dialogbox>
  </eicommands>
</oncodelineclick>

\end{lstlisting}

\medskip
\noindent
The \lst{highlightlines} indicates that lines 5--10 of the file
\texttt{/Examples\_1/iterative/sum.s} (which is opened in the editor)
should be highlighted. The \lst{oncodelineclick} tag indicates that
when clicking on line $17$, a dialog-box with a corresponding message
should be opened.
%
Note that the application is only modified once to produce such
output, and will have similar effect in all interfaces that support
this output language.

%%% Local Variables: 
%%% mode: latex
%%% TeX-master: "manual"
%%% End: 
