{% -*- mode: LaTeX; TeX-PDF-mode: t; TeX-master: "manual"; -*-
}

\chapter{The \ei Output Language}
\label{ch:eiol}

In this chapter we describe a text-based output language that allows
application to view their output in a graphical way, e.g.,
highlighting lines, adding markers, defining on-click actions, etc.
%
Some clients, e.g., the web-client, interpret this language and render
the effect of the corresponding commands in the respective environment.


\section{Brief description}
\label{outputmanager:sec:desc}

\section{Details XML output}
\label{outputmanager:sec:output}


%% 
\bigskip

%% EIOUT
%%
\xmlstruct{eiout} 
{
%
  This is the main environment of the output, it includes several list
  of commands environment \xmlstructref{eicommands}, and several list
  of actions actions \xmlstructref{eiactions}. 
%
  Commands are executed first, in the given order, and then actions are
  executed in the given order as well.
%
  The \xmlstructattr{version} attribute indicates the version of the
  output language that is used, which is $1.0$ by default.
%
}
{The \app response (see Section~\ref{outputmanager:sec:desc})}


%% EICOMMANDS
%% 
\bigskip
\xmlstruct
{eicommands}
{
%
A list of commands to be performed.
%
The attribute \xmlstructattr{dest} is the destination file on which
the command is applied (if needed).
%
E.g., when highlighting a line we might want to highlight a line in
one file or another. 
%
If \xmlstructattr{dest} is not specified, then the commands will be
applied to the file that is currently active, e.g., if the client have
a code editor with several tabs, one for each file, the command will
be applied to the active tab. If none is active then the behavior is
not specified.
% 
The attribute \xmlstructattr{outclass} specifies the \emph{output
  class} of the commands in this environment, that is, the nature of
the corresponding output generated by the commands, e.g., error,
information, warning, etc.
%
All commands inside this environment inherit the values of
\xmlstructattr{outclass} and \xmlstructattr{dest}, and each can
overwrite them.
%
}
{\xmlstructrefwb{eiout}, \xmlstructrefwb{onclickaction}, \xmlstructrefwb{oncodelineclickaction}}

%% EIACTIONS
%% 
\bigskip
\xmlstruct
{eiactions}
{
%
  A a list of actions to be declared. An action typically executes a
  list of \xmlstructref{eicommands} when the user interacts with the
  interface in some predetermined way, e.g., \emph{when the user
    clicks on line 30, highlight lines number 12 and 16}. We say the
  an action is \emph{performed} as a response to the user interaction.
%
  If the user interacts again with the interface, according to what is
  specified in the action, then the action is \emph{unperformed} if
  possible (when the corresponding commands support the \emph{undo}
  operation), e.g., in the above example if the user clicks again on
  line 30 the highlights of lines 12 and 16 are turned off.

  Before \emph{performing} an action, the last \emph{performed} action
  is \emph{unperformed} first. This behavior can be disabled by
  setting the \xmlstructattr{autoclean} attribute to
  \lst{``false''}. All actions inside this environment inherit the
  value of \xmlstructattr{autoclean}, and each can overwrite it.
% 
  The attribute \xmlstructattr{dest} and \xmlstructattr{outclass} are
  as in the case of commands (see the description of
  \xmlstructref{eicommands}).
%
}
{\xmlstructrefwb{eiout}}


%% EICOMMAND
%% 
\bigskip
\xmlstruct
{eicommand}
{
A command in the \ei output language, briefly:
\begin{itemize}
\item \xmlstructref{printonconsolecommand} can be used to print on the console.
%
\item \xmlstructref{highlightlinescommand} can be used to highlight
  lines in the code editor.
%
\item \xmlstructref{dialogboxcommand} can be used to open a dialog
  window with a corresponding message.
%
\item \xmlstructref{writefilecommand} can be used to add a file (and a
  corresponding content) to the files tree.
%
\item \xmlstructref{setcsscommand} can be used to change the CSS
  properties of some elements.
%
\item \xmlstructref{addmarkercommand} can be used to add a marker next
  to a line in the code editor.
%
\item \xmlstructref{addinlinemarkercommand} can be used to add a line
  widget (an inlined marker) in the code editor.
%
\end{itemize}
}
{\xmlstructrefwb{eicommands}}


%% EIACTION
%% 
\bigskip
\xmlstruct
{eiaction}
{
An action in the \ei output language, briefly:
\begin{itemize}
%
\item \xmlstructref{oncodelineclickaction} can be used to perform an action
  when the user clicks on a line in the code editor.
%
\item \xmlstructref{onclickaction} can be used to perform an action
  when the user clicks on a DOM element.

\end{itemize}
}
{\xmlstructrefwb{eiactions}}


%% PRINTONCONSOLE
%%
\bigskip
\xmlstruct
{printonconsolecommand}
{
%
  Print the content of the \xmlstructref{content} environments on the
  console with identifier \xmlstructattr{consoleid}.
%
  If \xmlstructattr{consoleid} is not specified, the output goes to
  the default console.
%
  If \xmlstructattr{consoleid} is specified but there is no console
  with such an identifier, the console is created and
  \xmlstructattr{consoletitle} (if specified) is used as its title.
%
  The attribute \xmlstructattr{outclass} is as described in
  \xmlstructref{eicommands}.
%
} 
{
\xmlstructrefwb{eicommand}
}

\bigskip
\xmlstruct
{highlightlinescommand}
{
%
  Highlight the lines specified by \xmlstructref{lines} in the file
  \xmlstructattr{dest}. The attribute \xmlstructattr{outclass} is as
  described in \xmlstructref{eicommands}.
%
}
{\xmlstructrefwb{eicommand}}

\bigskip
\xmlstruct
{dialogboxcommand}
{
%
  Open a dialog box with the content specified by the
  \xmlstructref{content} environemnts. The value of
  \xmlstructattr{boxtitle}, if specfied, is used as a title for the
  dialog box. The attribtes \xmlstructattr{boxwidth} and
  \xmlstructattr{boxheight} can be used to set the size of the window.
  The attribute \xmlstructattr{outclass} is as in
  \xmlstructref{eicommands}.
%
}
{\xmlstructrefwb{eicommand}}

\bigskip \xmlstruct {writefilecommand} 
{
%
  Create a new file and place it in the files view, using the path
  specified by \xmlstructattr{filename}. The file is initialized with
  the all text insider this tag. If the file exists, and
  \xmlstructattr{overwrite} is true the content is replaced otherwise
  a new file is created with a new name. The default value of
  \xmlstructattr{overwrite} is false.
% 
  This command does not support \emph{undo}.
%
}
{\xmlstructrefwb{eicommand}}

\bigskip
\xmlstruct
{setcsscommand}
{
%
  Change the CSS properties, as specified bt
  \xmlstructref{cssproperties}, of all elements that match the
  selector in \xmlstructref{elements}.
%
  There must be exactly one \xmlstructref{elements} environment and
  one \xmlstructref{cssproperties} environment. 
%
  The elements are typically selected from those generated by other
  commands.
%
}
{\xmlstructrefwb{eicommand}}

\bigskip
\xmlstruct
{addmarkercommand}
{
%
  Add a marker next to each line that is specified in
  \xmlstructref{lines}. The column information from each
  \xmlstructref{line} in \xmlstructref{lines} is ignored.  All markers
  are associated with the content given by the \xmlstructref{content}
  environments, as a tooltip. If the client allows expanding the
  tooltip to a dialog window, the the attributes
  \xmlstructattr{boxtitle}, \xmlstructattr{boxwidth} and
  \xmlstructattr{boxheight} can be used to set the properties of the
  corresponding window (see \xmlstructref{dialogbox}).
%
%as
  The attributes \xmlstructattr{dest} and \xmlstructattr{outclass} are
  as described in \xmlstructref{eicommands}.
%
}
{\xmlstructrefwb{eicommand}}

\bigskip
\xmlstruct
{addinlinemarkercommand}
{
%
  Add an inline marker (a line widget) for each line that is specified
  by \xmlstructref{lines}. All line widgets will include the content
  specified by the \xmlstructref{content} environments. In some
  clients, the supported content might be only of text format.
%
  The attributes \xmlstructattr{dest} and \xmlstructattr{outclass} are
  as described in \xmlstructref{eicommands}.
%
}
{\xmlstructrefwb{eicommand}}

\bigskip
\xmlstruct
{oncodelineclickaction}
{
%
  Add markers at the the code lines specified by \xmlstructref{lines},
  such that when any is clicked the commands in
  \xmlstructref{eicommands} are performed.
  % 
  The content given by the \xmlstructref{content} environemts is
  associated with the markers.
%
  The attributes \xmlstructattr{dest} and \xmlstructattr{outclass} are
  as described in \xmlstructref{eiactions}. Moreover, the above
  \xmlstructref{eicommands} environment inherits the
  \xmlstructattr{dest} and \xmlstructattr{outclass} attributes of this
  environment.
%
}
{\xmlstructrefwb{eiaction}}

\bigskip
\xmlstruct
{onclickaction}
{
%
  A click on any DOM element that matches the selector of
  \xmlstructref{elements}, will execute the commands declared in
  \xmlstructref{eicommands}.
%
  The attributes \xmlstructattr{dest} and \xmlstructattr{outclass} are
  as described in \xmlstructref{eiactions}. Moreover, the above
  \xmlstructref{eicommands} environment inherits the
  \xmlstructattr{dest} and \xmlstructattr{outclass} attributes of this
  environment.
%
}
{\xmlstructrefwb{eiaction}}


%% 
\bigskip
\xmlstruct
{lines}
{
A group of lines, typically used to specify the lines affected by 
an \xmlstructref{eicommand} or an \xmlstructref{eiaction}.
}
{\xmlstructrefwb{eicommand}, \xmlstructrefwb{eiaction}}

%% 
\bigskip
\xmlstruct
{line}
{
%
  A region (of lines) typically used to specify the region on which
  the effect of an \xmlstructrefwb{eicommand} or an
  \xmlstructrefwb{eiaction} is applied:
%
\begin{itemize}
\item \xmlstructattr{from} is the start line.
\item \xmlstructattr{to} is the end line.
\item \xmlstructattr{fromch} is the  where the first line starts.
\item \xmlstructattr{toch} is the character (i.e., column number) where the last line ends.
\end{itemize}
%
The default value of \xmlstructattr{to} is as the value of
\xmlstructattr{from}. The default value of \xmlstructattr{colfrom} is
0, and of \xmlstructattr{colto} is the end of the line.
%
} 
{\xmlstructrefwb{lines}}

\bigskip
\xmlstruct
{elements}
{
Set of elements
} 
{\xmlstructrefwb{setcsscommand}, \xmlstructrefwb{onclickaction}}

\bigskip
\xmlstruct
{selector}
{
%
  The attribute \xmlstructattr{value} musy be valid selector as in
  JQuery. It is used to match some DOM elements.
%
}
{\xmlstructrefwb{elements}}

\bigskip
\xmlstruct
{cssproperties}
{
%
A set of CSS properties.
%
}
{\xmlstructrefwb{setcsscommand}}

\bigskip
\xmlstruct
{cssproperty}
{
%
  A CSS property. The attributes \xmlstructattr{name} and this
  \xmlstructattr{value} should correspond to valid CSS properties.
%
}
{\xmlstructrefwb{properties}}


\noindent
\xmlstructdef{consoleid}

( [a-z,A-Z,0-9,-,\_]+ | new | default )

The value 'new' means a new console, no reference to this console is
saved. The value 'default' means the default console of the client.

\noindent
\xmlstructdef{path}

a path to file, including the file name, starting from the root /\_ei\_files/

\noindent
\xmlstructdef{version}

$x.y$, where $x$ is the major version number and $y$ is the minor one,
e.g. $1.0$, $1.1$, etc.

\noindent
\xmlstructdef{outclass}

none | info | warning | error 


