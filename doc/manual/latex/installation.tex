{% -*- mode: LaTeX; TeX-PDF-mode: t; TeX-master: "manual"; -*-
}

\chapter{Installation Guide}
\label{ch:installation}

In this appendix we explain how to install the server and how to use the
different clients.

\section{Downloading \ei}
\label{ch:installation:downloading}

We assume that you have already downloaded%
\footnote{\url{http://github.com/abstools/easyinterface}} 
%
\ei into a directory called \lst{easyinterface}, and that all files
inside this directory have read and execute permissions to
\lst{others} which can be done, for example, in Unix based systems
by executing:

\medskip
\begin{lstlisting}
> chmod -R 755 easyinterface
\end{lstlisting}

\medskip
\noindent
The purpose of this is to make all files visible to the
\lst{Apache\ Web\ Server} on which the \ei server runs.
This is (most likley) not required if you are using Microsoft Windows.

\section{Installing \ei Server}
\label{ch:installation:installing-server}

The installation consists in installing an \lst{Apache\ Web\ Server}
(with \lst{PHP} > 5.0 enabled) and then configuring it to recognize the
\lst{easyinterface} directory. If you already have Apache installed
and the \lst{easyinterface} directory is visible then no further
configuration is required, simply visit the corresponding address (e.g.,
if it is placed in the \lst{public_html} directory, visit
\url{http://localhost/~user/easyinterface}). Otherwise follow the steps
below depending on which operating system you are using, \lst{Linux},
\lst{OS X} or \lst{Windows}.

\subsection{Linux}\label{ch:installation:installing-server:linux}

Installing Apache depends on the Linux distribution you are using, for
example, if you are using Ubuntu you can install it by executing the
following in a shell:

\medskip
\begin{lstlisting}
> sudo apt-get update
> sudo apt-get install apache2
> sudo apt-get install php5 libapache2-mod-php5 php5-mcrypt
> sudo service apache2 start
\end{lstlisting}

\medskip
\noindent
Once installed test that it works correctly by visiting
\url{http://localhost} and test that PHP works correctly by visiting
\url{http://localhost/info.php} (this address might be different from
one distribution to another). Next, to make the \lst{easyinterface}
directory visible, edit \lst{/etc/apache2/mods-enabled/alias.conf}
and add the following lines:

\medskip
\begin{lstlisting}
  Alias /ei "/path-to/easyinterface"

  <Directory "/path-to/easyinterface">
     Options FollowSymlinks MultiViews Indexes IncludesNoExec
     AllowOverride All
     Require all granted
  </Directory>
\end{lstlisting}
\medskip
\noindent
To activate this change you need to restart Apache by executing the
following in a shell:
\medskip
\begin{lstlisting}
> sudo service apache2 restart
\end{lstlisting}

\medskip
\noindent
Now visit \url{http://localhost/ei} to check that \ei works
correctly. If no error message is shown, you can proceed to the next
section and start using the Web client.

\subsection{OS X}\label{ch:installation:installing-server:os-x}

OS X typically comes with Apache installed, and all you need is to
configure it to recognize the \lst{easyinterface} directory. To do
so, edit \lst{/etc/apache2/httpd.conf} add the following lines:

\medskip
\begin{lstlisting}
  Alias /ei "/path-to/easyinterface"

  <Directory "/path-to/easyinterface">
     Options FollowSymlinks MultiViews Indexes IncludesNoExec
     AllowOverride All
     Require all granted
  </Directory>
\end{lstlisting}

\medskip
\noindent
To activate this change you need to restart Apache by executing the
following in a shell:

\medskip
\begin{lstlisting}
> sudo apachectl restart
\end{lstlisting}

\medskip
\noindent
You can also restart Apache using
\lst{System Preferences > Sharing > Web Sharing}.
Now visit \url{http://localhost/ei} to check that \ei works
correctly.

\subsection{Microsoft Windows}\label{ch:installation:installing-server:windows}

Apache Web Server for Microsoft Windows is available from a number of
third party vendors.%
\footnote{\url{http://httpd.apache.org/docs/current/platform/windows.html\#down}}
We have tested \ei using WampServer.%
\footnote{\url{http://www.wampserver.com/}}

Install the WampServer, for example in \lst{c:\wamp}, and then edit
the configuration file
\lst{c:\wamp\bin\apache\apache.X.Y.Z\httpd.conf} and add the following
lines to make the \lst{easyinterface} directory visible:

\medskip
\begin{lstlisting}
  Alias /ei "\path-to\easyinterface"

  <Directory "\path-to\easyinterface">
     Options FollowSymlinks MultiViews Indexes IncludesNoExec
     AllowOverride All
     Require all granted
  </Directory>
\end{lstlisting}

\medskip
\noindent
Next restart the WampServer by executing

\medskip
\begin{lstlisting}
c:\wamp\wampserver.exe -restart
\end{lstlisting}

\medskip
\noindent
Now visit \url{http://localhost/ei} to check that \ei works
correctly. If you have permission problems when accessing this address,
try to remove the file \lst{easyinterface/.htaccess}.

By default the server is configured to execute the demo applications
in a Unix based operating system, for using them in Windows you should
copy the configuration file
\lst{$\mbox{server/config/eiserver.default.win.cfg}$} to
\lst{$\mbox{server/config/eiserver.cfg}$}.

The demo applications are simple bash scripts, and thus you need to
install \href{http://win-bash.sourceforge.net/}{win-bash} if you want to
use them. To do so, simply download the corresponding \lst{zip} file
and extract it in \lst{c:\bash} (it is important to
place it in \lst{c:\bash} since the configuration
files use \lst{c:\bash\bash.exe} to
execute the bash scripts).

\section{Installing and Using \ei Clients}
\label{ch:installation:installing-clients}

The Web client can be used by visiting
\url{http://localhost/ei/clients/web}. 

