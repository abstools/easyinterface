{% -*- mode: LaTeX; TeX-PDF-mode: t; TeX-master: "manual"; -*-
}

\chapter{Introduction}

During the lifetime of a research project, different partners
typically develop several research prototype tools that share many
common aspects.
%
For example, in the \envisage~\cite{envisage} project, in the context
of which this work was developed, several tools for processing \abs
programs~\cite{johnsen10fmco} have been developed: static analyzers,
test-case generators, compilers, simulators, etc.
%
This observation is equally true for researchers as individuals and as
groups: during a period of time they develop several related tools to
pursue a specific research line. For example, in the
\costa\footnote{\url{http://costa.ls.fi.upm.es}} group, we have
developed many program analysis and test-case generation tools for
several programming languages and abstract models.

Making research prototype tools available to the corresponding
research communities is of utmost importance for several reasons: to
promote the corresponding research; to make the community aware of the
corresponding project or research group; to get valuable feedback; to
increase the tools' lifetime beyond the duration of a specific
project; etc. However, making tools available is not enough, they
should also be easy to use since otherwise users would avoid using
them, in particular if they require some technical skills to install
or use them.
%
One way to achieve this is to build graphical user interfaces (GUIs)
that facilitate using the tools; in particular, web-interfaces in
order to avoid the overhead/risk of downloading and installing the
tools locally.

Building GUIs is a tedious task, in particular if they have to be
developed from scratch. Building web-interfaces is even more difficult
since web-based applications are typically more difficult to debug.
%
Thus, the task of building GUIs for research prototype tools typically
gets a lower priority in the development process, and the effort is
instead directed to improving the functionality of the tools, rather
than their corresponding GUIs. In addition, due to these difficulties,
researchers often opt for copying the GUI of one tool and modifying it
to fit the needs of a new related tool.
%
Apart from code duplication, these tools will ``live'' separately,
even though they might benefit from having them all in a common
environment since they are related and, for example, a user who is
interested in one tool will be easily exposed to other related tools
when accessing the common environment.

Clearly many of the difficulties, and the corresponding effort, in
building GUIs are unavoidable in general.  This is because
%
\begin{inparaenum}[\upshape(\itshape i\upshape)]
%
\item tools produce different outputs that are presented graphically
  in different ways to the user; and
%
\item the input --- and the interaction with the user --- is different
  from one tool to another.
%
\end{inparaenum}
%
However, if we consider a set of related tools, e.g., those developed
in a given research project, it is easy to identify many common
aspects in their input and output. These common aspects can be then
used as a bases for building a GUI construction toolkit that provide
%
\begin{inparaenum}[\upshape(\itshape i\upshape)]
%
\item an easy way to integrate tools in a common environment; and 
%
\item an easy way to present the output using a predefined set of
  graphical widgets that cover the common aspects of the output.
%
\end{inparaenum}
%
Despite of the fact that such a toolkit is limited when compared to
general GUI libraries, in the sense that it can be mostly used for
tools that fit in the identified common aspects, researches would
prefer this approach if it extremely simplifies the task of building a
GUI for a new tool, e.g., if it allows building such a GUI in few
hours and without the need for deep modifications to the corresponding
code.
%
Building such a toolkit is the main objective of this work.

\section{Objectives}
\label{sec:intro:objectives}

The main objective of this work is to develop a toolkit that can be
used to easily develop GUIs for research prototype tools, and,
moreover, integrating them in a common environment. We focus on tools
that have the following common aspects:
%
\begin{itemize}
%
\item A request to a tool corresponds to executing a program from a
  command-line, where the input is passed as command-line parameters
  (including file names) and the output is printed on the standard
  output. Note that this does not limit us to command-line tools,
  since, for example, if the tool runs as a server we can write a
  command-line wrapper that forwards the requests to the server and
  prints the server's response on the standard output.
%
\item The tools are, in principle, supposed to process programs, e.g.,
  static analysis tools. Thus, they receive as parameters
\begin{inparaenum}[\upshape(\itshape i\upshape)]
\item file names that represent programs; and
\item possibly a list of program entities (e.g., method names, class
  names) which indicate some information to the tool (e.g., for static
  analysis tools it might indicate where to start the analysis from).
\end{inparaenum}
%
\item The output of the tool mainly includes information that is
  related to some parts of the input program. For example, associating
  information to a program line, drawing graphs that represent some
  property of the program such as resource consumption, etc.
\end{itemize}
%
Note that the above aspects where not chosen arbitrarily, but they
rather cover tools developed in the \envisage~\cite{envisage} project,
in the context of which this work was developed, and the tools (and
research lines) of the \costa group to which the author of this work
belongs. Later, the reader will see that our toolkit is not actually
limited to tools that satisfy the above conditions, but rather it is
more general.

As regards the capabilities of the toolkit that we want to develop, we
aim at developing one that complies with the following objectives:
%
\begin{itemize}
%
\item[\objdef{simplegui}] Using the toolkit for developing
  \emph{simple} GUIs for existing tools, and integrating them in a
  common environment, should not take more than few minutes, and,
  moreover, \emph{without} requiring any modification to the tool's
  code at all.
%
\item[\objdef{complexgui}] Developing more sophisticated GUIs might
  require mild modifications to the tool's code, and by no mean deep
  modifications --- requiring deep modifications would make it less
  likely that the toolkit will be used.
%
\item[\objdef{outputlang}] Using the toolkit for presenting the output
  graphically should not require any knowledge on GUI or WEB
  programming. The user should describe the output in a natural
  language, e.g., ``\emph{highlight line number 10 of file ex.c}'' (or
  a structured version of such statement, e.g., using XML or JSON, to
  be able to parse it easily).
%
\item[\objdef{commonenv}] The toolkit must provide common environments
  in which tools can be integrated. The most important environment is
  the web-based one (i.e., it runs in a web browser). In addition, the
  design should allow developing more environments in the future
  without modifying the toolkit or existing integrated tools.
%
\item[\objdef{commonenvtrans}] The common environments should be
  completely transparent to the integrated tools. This means that the
  work for integrating the tool or modifying it to produce graphical
  output should be done only once, and it should work in all
  environments equally (including future ones).
%
\item[\objdef{security}] The toolkit should be secure, it should not
  pose any security risks both to developers and users of to the
  corresponding tools.
  % 
\end{itemize}
%
To evaluate the developed toolkit, it should be used to build GUIs for
the tools of the \envisage project, and, moreover, integrating them in
a common environment.

\section{Contributions}

In this work we have developed \ei, a toolkit for easily building
GUIs for research prototype tools that comply with the objectives and
requirements stated in Section~\ref{sec:intro:objectives}. 
%
In particular, we developed a new methodology for building GUIs for
research prototype tools that consists of the following:

\begin{itemize}

\item A server side where tools are installed. Adding a tool to the
  server is done by adding a configuration file (which is very easy to
  write) describing how to run it, etc.

\item A protocol that allows connecting to the server for, among many
  other things that we will see later, executing a tool on a
  particular input and getting back the result. In some sense, this
  allows converting the tools installed on the server to services.

\item A web-based development environment that allows users to use the
  tools installed on an \ei servers transparently, in addition to
  the functionality of a normal development environment.
  
\item An output text-based language that can be used to describe how
  the output should be shown graphically, and corresponding
  interpreters in the development environments to convert these
  descriptions to graphical effects.

\item An extensive evaluation by using \ei to integrate the tools of
  the \envisage project in a common development environment.

\end{itemize}
%
\ei is open source and available at GitHub\footnote{\eigithub}.

\section{The Structure of This Document}

This document is structured as follows. 
%
In Chapter~\ref{ch:architecture} we describe the overall architecture
of \ei and how its different components interact.
%
In Chapter~\ref{ch:quickguide} we describe how to integrate a new
tool in \ei in details. The purpose of this chapter is to allow the
reader to briefly get familiar with the concrete details before moving
on to the next chapters.
%
In Chapter~\ref{ch:server} we describe the \ei server specifications,
and a corresponding implementation.
%
In Chapter~\ref{ch:clients} we describe the available clients
(web-interface, etc). 
%
In Chapter~\ref{ch:eiol} we describe the syntax and semantics of the
\ei output language. 
%
In Chapter~\ref{ch:evaluation} we describe the evaluation that we have
done in the context of the \envisage project.
%
Finally, in Chapter~\ref{ch:conclusions}, we conclude and discuss
future and related work.
%
In addition, we include the installation guide of \ei as
Appendix~\ref{ch:installation}.




