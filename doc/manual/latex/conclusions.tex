{% -*- mode: LaTeX; TeX-PDF-mode: t; TeX-master: "manual"; -*-
}

\chapter{Conclusions, Related and Future Work}
\label{ch:conclusions}


In this work, we have addressed the problem of \emph{easily
  constructing} GUIs for research prototype tools, and integrating
them in common environments. This is crucial since
%
\begin{inparaenum}[\upshape(\itshape i\upshape)]
%
\item it reduces the effort dedicated to building GUIs, which is
  usually tedious and complicated, and thus the main effort can be
  dedicated to improving the functionality of the tools; and
%
\item it makes the tools continuously available to corresponding
  research communities since modifying the GUI when a tool changes, if
  needed, is immediate, and thus the dissemination of the
  corresponding research is improved as well --- note that research
  prototype tools are expected to change continuously, for example,
  during a research project.
%
\end{inparaenum}

Clearly, attempting to significantly reduce the effort required for
building GUIs in general is not feasible, since tools produce
different outputs and receive different input. However, as we have
observed in this work, this become feasible when focusing on a set
of related tools that have common input and output aspects. In this
work, we have focused on the tools developed in the
\envisage~\cite{envisage} project which include: static analyzers,
test-case generators, compilers, simulators, etc. We have developed a
toolkit called \ei that \emph{extremely simplifies} the way GUIs are
constructed for such tools, and the way they are integrated in common
environments as well.


\ei is a toolkit that consists of two main components:
%
\begin{inparaenum}[(1)]
%
\item a \emph{server} where tools can be installed by providing simple
  configuration files, and then can be accessed as services using some
  protocol that we have defined; and
%
\item a \emph{client}, in the form of a web-based development
  environment, that makes it easy to communicate with the server side
  to execute a tool.
%
\end{inparaenum}
%
An important feature of the design of \ei is that once a tool is
installed on an \ei server, it will automatically appear in all \ei
clients that connect to this server, without any additional
effort. Thus, this design allows installing new tools in a common
environment with minimal effort, i.e., in few minutes.

Another important outcome of this work is the \ei output
language. It is text-based language that tools can use to present
their output graphically, if the corresponding client support it.
%
Importantly, this language does not require any knowledge on GUI or
Web programming. The advantage of using this output language is that
it is interpreted by all \ei clients equally. Thus, the tool is
modified once to use this language and the effect will take place in
all clients (including ones that will be developed in the future).

The \ei toolkit has been successfully used in the context of the
\envisage project, where the tools developed by the different partners
has been integrated in a common web-based environment. Some of the
tools use the \ei language as well, and others, whose output is very
simple, do not.

\section{Future Work}

We have identified several future work directions, that would make \ei
a more powerful toolkit for building GUIs for research prototypes:

\begin{itemize}

\item Generating output in the \ei output language is currently done
  by directly printing the corresponding XML on the standard output.
  %
  It would be useful to provide libraries, for different programming
  languages, that abstract this level away. Namely, tools will not
  print directly but rather would use methods/objects of such
  libraries to generate the output, which in turn will generate the
  corresponding XML.
  %
  Note that we have such a library for Prolog, which we use in tools
  marked by \lst{SACO} in Section~\ref{sec:evaluation:envisage}.

\item Developing more \ei clients, as the ones describes in
  Section~\ref{sec:clients:other}, would make the toolkit adequate for
  different scenarios. For example, an Eclipse plugin would allow
  users to continue using their preferred development environment
  instead of learning to use a new one.

\item Developing a simpler web-client that allows to inline an \ei
  environment inside HTML documents. This is useful for writing
  interactive tutorials for example. Currently we have a simple
  support for this as described in
  Section~\ref{sec:clients:web:other}.

\item Improving the handling of sessions (in the \ei server) to
  provide tools with and easy way to store and reload sessions. To
  take full advantage of this feature, the \ei output language should
  be also extended to allow new kind of interaction that allows
  specialized callbacks to the corresponding tools.

\end{itemize}

\noindent
We plan to continue the work on \ei in the near future following the
above directions, independently from the \envisage project.

\section{Related Work}

There are many powerful web-based IDEs that allow developers to
develop their code online, and, in addition, some can be customized to
connect external tools. In this section we overview some closely
related ones.

\lst{Orion}~\cite{orion} is a web-based IDE developed by the Eclipse
Foundation.\footnote{\url{http://www.eclipse.org}}
%
It provides some powerful features like connecting to \texttt{git}
repositories, syntax-highlighting, etc. Once can use it to develop and
compile code in several programming languages like C, C++ or Java.
%
It also includes some plug-ins that can be activated like
\emph{web-tools support} for editing \lst{HTML} and \lst{CSS},
\emph{JSON editor} for editing \lst{JSON} records, etc.

%% Ground Coding
%%
\lst{Coding Ground}~\cite{codingground} is a web-based IDE developed
by TutorialsPoint.\footnote{\url{http://www.tutorialspoint.com}}
%
It was developed with the main objective of adding exercises to their
tutorials, but since then it has evolved into an IDE where developers
can edit, compile, execute, and share their code.
%
It supports more than 90 programming languages, but only one at a
time. It does not allow integrating external tools.

%% Cloud9
%%
\lst{Cloud9}~\cite{cloud9} is a web-based IDE that supports hundreds
of programming languages, and allows creating collaborative workspaces
with multiple real-time edition.
%
It has some advanced features like code completion, name refactoring,
etc. It also provides support for using repositories like
\texttt{git}, \texttt{mercurial} and \texttt{FTP} servers.
%

%% CODEBOARD
%
\lst{Codeboard}~\cite{codeboard} is web-based IDE to teach programming
in the classroom. One can easily create and share exercises with
students, analyze and inspect students' submissions, etc. One can
customize it to connect external tools, however, their output can be
shown only on the console area.


The IDEs described above are very powerful, but focused on developing
code. They do not address most of the objectives that we stated in
Section~\ref{sec:intro:objectives}, in particular
%
\begin{inparaenum}[\upshape(\itshape i\upshape)]
\item they do not provide an easy way to integrate external tools; and
\item they do not provide a simple way to produce output or code
  annotations as in our output language that is described in
  Chapter~\ref{ch:eiol}.
\end{inparaenum}

%% RISE4FUN
%
An exception is the tool \lst{rise4fun}~\cite{rise4fun}, which is
developed by Microsoft to allow making, among others, program analysis
tools available online.
%
However, it is far simpler than \ei: 
\begin{inparaenum}[\upshape(\itshape i\upshape)]
\item tools are not integrated in a common environment, but rather
  each has its own page; 
\item the output can be shown only in a console area; and
\item tools cannot easily receive parameters.
\end{inparaenum}



