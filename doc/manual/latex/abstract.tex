{% -*- mode: LaTeX; TeX-PDF-mode: t; TeX-master: "manual"; -*-
}

\newpage

\thispagestyle{empty}

\begin{center}
{\bf \Huge Abstract}
\end{center}

\vspace{1cm}
During the lifetime of a research project, different partners develop
several research prototype tools that share many common aspects.
%
This is equally true for researchers as individuals and as groups:
during a period of time they often develop several related tools to
pursue a specific research line.
%
Making research prototype tools easily accessible to the community is
of utmost importance to promote the corresponding research, get
feedback, and increase the tools' lifetime beyond the duration of a
specific project.
%
One way to achieve this is to build graphical user interfaces (GUIs)
that facilitate trying tools; in particular, with web-interfaces one
avoids the overhead of downloading and installing the tools.

Building GUIs from scratch is a tedious task, in particular for
web-interfaces, and thus it typically gets low priority when
developing a research prototype. Often we opt for copying the GUI of
one tool and modifying it to fit the needs of a new related tool.
%
Apart from code duplication, these tools will ``live'' separately,
even though we might benefit from having them all in a common
environment since they are related.

This work aims at simplifying the process of building GUIs for
research prototypes tools. In particular, we present \ei, a toolkit
that is based on novel methodology that provides an easy way to make
research prototype tools available via common different environments
such as a web-interface, within Eclipse, etc.
%
It includes a novel text-based output language that allows to present
results graphically without requiring any knowledge in GUI/Web
programming. For example, an output of a tool could be (a structured
version of) ``\emph{highlight line number 10 of file ex.c}'' and
``\emph{when the user clicks on line 10, open a dialog box with the
  text ...}''. The environment will interpret this output and converts
it to corresponding visual effects. The advantage of using this
approach is that it will be interpreted equally by all environments of
\ei, e.g., the web-interface, the Eclipse plugin, etc.
%

\ei has been developed in the context of the \envisage~\cite{envisage}
project, and has been evaluated on tools developed in this project,
which include static analyzers, test-case generators, compilers,
simulators, etc. \ei is open source and available at
GitHub\footnote{\eigithub}.


\vspace{1cm}


\begin{center}
{\bf \Large Keywords}
\end{center}

\vspace{0.5cm}
   
Generic User Interfaces, Web Programming.

   

